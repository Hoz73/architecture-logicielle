\documentclass[a4paper,10pt]{article}
\usepackage[utf8]{inputenc}
\usepackage{listings}
\usepackage{graphicx}
\usepackage{subcaption}
\usepackage{adjustbox}
\usepackage{hyperref}
\usepackage{imakeidx}

\title{Compte rendu TP : Architecture logicielle}
\date{29/05/21}
\author{Hamze Al-Rasheed - Nicolas Commandeur - Benjamin Verdant - Robin Wagner}

\begin{document}
    \maketitle
    \pagenumbering{gobble}
    \newpage
    \tableofcontents
    \newpage
    \pagenumbering{arabic}
    
    \section{Sujet}
        Le principe du tp est de concevoir l'architecture d'un système de contrôle d'accès à un ensemble de bâtiments.
    
        Un batiment possède un nom ainsi que des informations. Les informations sont :
        \begin{itemize}
            \item la liste des portes du bâtiment
            \begin{itemize}
                \item Le nom de la porte    \newline
                \textit{ex : Porte sud, 8A-44, etc}
                \item L'id de la badgeuse d'entrée et de sortie de la porte
                \newline
                Impair pour entrer et pair pour sortir \newline
                \textit{ex : 11 pour entrer dans le bâtiment 1 et 12 pour en sortir}
                \item La liste des cartes autorisées ou non 
                \newline
                \textit{ex : Badgeuse 11 : carte 1 autorisée, carte 2 non autorisée, etc}
                \item L'état de la porte
                \newline
                \textit{ex : Ouvert ou fermé}
            \end{itemize}
        \end{itemize}
        
        Les utilisateurs ont chacun une \textbf{carte} qui a un \textbf{id} unique et le nom du détenteur de celle-ci.
        \newline
        \textit{ex : carte 1 $\Rightarrow$ Livai, carte 2 $\Rightarrow$ Eren }

        Pour accéder à un bâtiment ou à une salle, un utilisateur doit poser sa carte sur la badgeuse.
        \newline
        \par Si la personne \underline{est autorisée} à entrer dans le bâtiment alors, la porte s'ouvre et une lumière verte s'affiche pendant \textbf{15 secondes} et la porte reste ouverte pendant le \textbf{même temps}.
        \newline
        Si la personne n'est pas autorisée à entrer dans le bâtiment, une lumière rouge s'allume sur celle-ci.
        \newline
        \par
        Quand une personne ouvre une porte, un laser se situe juste après celle-ci, pour compter le nombre de personne qui passe. Plusieurs cas possibles :
        \begin{itemize}
            \item Une seule personne passe $\Rightarrow$ la personne qui a ouvert la porte est enregistrée dans le bâtiment et une trace de son passage est inscrit dans le log des passages.
            \item Plusieurs personnes passent $\Rightarrow$ une alarme retentit
            \item Personne ne passe $\Rightarrow$ rien ne se passe
        \end{itemize}

        Dans tous les cas, la porte se ferme au bout de \textbf{15 secondes}
        \newline

        \par En cas d'un incendie, toutes les portes sont débloquées et on inscrit dans un fichier toutes les personnes dans les bâtiments.
        \pagebreak
    \section{Prémice}
    \pagebreak
    \section{Architecture utilisée}
    \pagebreak
    \section{Choix technologiques}
    \pagebreak
    \section{Petit pas pour l'homme}
    \pagebreak
    \section{Grand pas pour l'humanité}
    \pagebreak
    \section{Difficultés rencontrées}
    \pagebreak
    
\end{document}